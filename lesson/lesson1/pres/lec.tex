\documentclass[12pt]{article}
\usepackage[noxcolor]{beamerarticle}
% Графика ----------------------------
\usepackage{graphicx} 
% Математика -------------------------
\usepackage{cmap} 
\usepackage{amsmath}
% Русский язык -----------------------
\usepackage[utf8]{inputenc} 
\usepackage[T2A]{fontenc}
\usepackage[russian]{babel}

% Для включение определенной лекции
\includeonlylecture{lec1}


% Размер сторон:
% ширина – 128 мм, высота – 96 мм (соотношение сторон 4:3).
\title{Искусственные нейронные сети}
\author[Даниил Копылов]{Копылов Д.Е., Михайлов А.А.}
% \date{}
\institute[ИДСТУ СО РАН, ИСП РАН, ИМИТ ИГУ]{
\inst{1}Институт динамики систем и теории управления им. В.М. Матросова Сибирского отделения Российской академии наук \and
\inst{2}Институт системного программирования им. В.П. Иванникова \\Российской академии наук \and
\inst{3}Институт математики и информационных технологий \\Иркутский государственный университет
}

%\insertauthor, \insertdate, \insertinstitute, \inserttitle, \insertsubtitle и \inserttitlegraphic

\begin{document}
\begin{frame}
    \maketitle
\end{frame}

\lecture[Задача классификации]{Задача классификации}{lec2}
\begin{frame}{Лекция 2}
\insertlecture
% \insertshortlecture %короткое название лекции
\end{frame}
\begin{frame}{формальная постановка задачи}
    Пусть есть два множества $X$ - множества объектов, $Y$ - множество ответов и предполагается,
    что существует функциональная зависимость 
    \begin{equation}
        f:X \to Y
    \end{equation}
\end{frame}
\begin{frame}
    \url{http://intsys.msu.ru/staff/mironov/machine_learning_vol1.pdf}
    \url{}
\end{frame}
\end{document}