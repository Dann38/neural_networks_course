\documentclass[]{beamer}

% Графика ----------------------------
\usepackage{graphicx} 
% Математика -------------------------
\usepackage{cmap} 
\usepackage{amsmath}
% Русский язык -----------------------
\usepackage[utf8]{inputenc} 
\usepackage[T2A]{fontenc}
\usepackage[russian]{babel}
\usepackage{graphicx}% Для картинки
\usepackage{subcaption}%Для картинки
\usepackage{float} %Для картинки
\usepackage{wrapfig}%Для оптикания картинок
\ifxetex
\usepackage{fontspec}
\setmainfont[
        ItalicFont=TimesNewRoman-it.ttf,
        BoldFont=TimesNewRoman-bold.ttf,
        Ligatures=TeX
]{TimesNewRoman.ttf}
\fi
% Для анимации -----------------------
\usepackage{animate}
\usepackage{tikz}

% Литература -------------------------
\usepackage[style=ieee]{biblatex}
\setbeamertemplate{bibliography item}{\insertbiblabel}
\addbibresource{main.bib}

% For more themes, color themes and font themes, see:
% http://deic.uab.es/~iblanes/beamer_gallery/index_by_theme.html
%
\mode<presentation>
{
  \usetheme{Madrid}
  %\usecolortheme{spruce}
  \usefonttheme{serif}    % or try default, structurebold, ...

} 

% On Overleaf, these lines give you sharper preview images.
% You might want to `comment them out before you export, though.
\usepackage{pgfpages}
\pgfpagesuselayout{resize to}[%
  physical paper width=8in, physical paper height=6in]

% Here's where the presentation starts, with the info for the title slide
\title[Тервер в МО]{Теория вероятностей в машинном обучение}
\author[Копылов Даниил]{Копылов Д. Е.}
\date{25 июля 2024 года}
\begin{document}

\begin{frame}
  \titlepage
\end{frame}

% \begin{frame}{Содержание}
%  \tableofcontents
% \end{frame}
\begin{frame}{События и их вероятность}
	\begin{block}{События}
		Для теории вероятности мир событиен. Событие одно из базовое понятие в теории вероятностей.
		События обозначают заглавными буквами английского алфавита $A, \; B\; ,...,Z$.
	\end{block}
	\begin{block}{Вероятность события}
		Вероятность наступления события обозначается $P(A)$ и характеризуется величиной из отрезка $[0, 1]$, 
		где 0 - событие никогда не наступит, 1 - событие обязательно наступит.
	\end{block}
	
	\begin{block}{Пример событий}
		$A$ - Идет дождь;

		$B$ - Ребенок вернулся с прогулки в 16:30

		$C$ - Модель предсказывает результаты с точностью выше 90\%
	
	\end{block}
\end{frame}
\begin{frame}{Условная вероятность и формула Байесса}
	Практически всегда события осуществляются совместно с другими событиями, и часто нелязя перестать наблюдать совместные события.
	Например $P(\text{A и B})$,  $A$ - ребенок вернулся с прогулки в 16:30, $B$ - дождь начался в 16:15.
	
	Иногда бывает интересно рассмотреть как будет вести себя одна величина, при определенном значение другой. Такая вероятность называется 
	условной и обозначается:
	$$P(A|B) = \frac{P(\text{A и  B})}{P(B)}$$

	Замечая особенность:
	$$P(A|B)\cdot P(B) = P(\text{A и B})$$
	$$P(B|A)\cdot P(A) = P(\text{A и B})$$
	$$P(A|B)\cdot P(B) = P(B|A) \cdot P(A)$$
	\begin{equation}
		P(A|B) = \frac{P(B|A) \cdot P(A)}{P(B)}
	\end{equation}
\end{frame}
\begin{frame}{Совместные и несовместные события}
	\begin{block}{}
		Событие несовместные, если $P(\text{A и B}) = P(A)P(B)$, т.е. $P(A|B) = P(A)$, $P(B|A) = P(B)$
	\end{block}
	\begin{block}{Для совместных событий операция "или"}
		Выполнение хотя бы одного из событий ("или"):
		$$P(\text{A или B}) = P(A) + P(B) - P(\text{A и B})$$
	\end{block}
	\begin{block}{Для несовместных событий операция "или"}
		Выполнение хотя бы одного из событий ("или"):
		$$P(\text{A или B}) = P(A) + P(B)$$
	\end{block}

\end{frame}
\begin{frame}{Случайные величины}
	\begin{block}{Не ясность}
		Иногда бывает, что ситуацию приходится описывать сразу большим множеством событий. 

		$A_1$ -- дождь начнется в 0:00,

		$A_2$ - дождь начнется в 0:01,

		...,

		$A_{1440}$ - дождь начнется в 23:59.
		
		
		Упростить себе жизнь можно введя параметр $\xi$. Теперь можно сказать:

		"Дождь начнется через $\xi$ минут после наступления полночи".

	\end{block}
	\begin{block}{Случайная величина}
		В таком случае случайное не событие (которое наступило или нет), а величина.
	\end{block}
\end{frame}
\begin{frame}{}
	\begin{block}{}
		Если вероятность $P(A_i) = p_i, \quad i = 1, 2,..., 1440$, 

		то вероятность $P(\xi = i) = p_i, \quad i = 1, 2,..., 1440.$

		
		Теперь мы говорим об вероятностях случайной величены, а не о вероятносте наступления события
	\end{block}

	\begin{block}{}
		В реальносте же можно говорить об вещественных числах, тогда событие невозможно описать 
		просто перечислев варианты: Температура воздуха будет $\xi$ градусов (без округления).
	\end{block}

	\begin{block}{}
		 Ястно, что температура \textit{почти наверное никогда}, не будет равняться 17 градусам ровно.
		 В случае вещественных чисел говорят об плотносте распределения величины и интервале в котором она может лежать. 
	\end{block}
\end{frame}

\begin{frame}{Функция плотности вероятности в случае вещественной случайной величины}	
	\begin{block}{}
	\begin{equation}
		P(\xi \in (\omega_1, \omega_2)) = \int_{\omega_1}^{\omega_2} p(\xi) d \xi ,
	\end{equation}
	\begin{equation}
		\int_{-\infty}^{+\infty} p(\xi) d\xi = 1, \quad  p(\xi) \geq 0
	\end{equation}
	$p(\xi)$ - функция плотности распределения.
	\end{block}
        Когда говорим о дискретных величинах с индексами из $I$, требуется выполнения следующих условий:
	$$\sum_{i \in I} p_i = 1, \quad p_i \geq 0 $$
\end{frame}

\begin{frame}{Функции от случайных величин}
	Со случайными величинами можно совершать привычные операции, только ответ также становится случайной величиной.

	$$\eta = f(\xi)$$

	Для того, чтоб можно было оценивать и случайные функции, вводятся понятие "математическое ожидание":
	\begin{equation}
		\mathbb{E}_{\xi \sim P} f(\xi) = \int_{-\infty}^{+\infty} p(\xi)f(\xi)d\xi
	\end{equation}

	Для дискретного случая имеет место понятие среднее значение:
	\begin{equation}
		\mathbb{E}_{\xi \sim P} f(\xi) = \frac{1}{n} \sum_{i=1}^{n} p(\xi)f(\xi)
	\end{equation}
\end{frame}

\begin{frame}[allowframebreaks]{bibliography}
% \printbibliography
\printbibliography
\end{frame}
\end{document}
